% !TEX root = Bachelorarbeit Synthetische Daten.tex
\chapter{Einleitung} \label{ch:intro}

Es folgt zunächst eine Einleitung in das Thema der Arbeit, die Motivation für die Untersuchung, die Zielsetzung und der Aufbau der Arbeit.

\section{Motivation} \label{sec:motivation}

Die Forschung in den Bereichen der Künstlichen Intelligenz (KI) und des Maschinellen Lernens (ML) deckt eine immer breitere Palette von Anwendungen ab, die an vielen Stellen in unserem Alltag Einzug halten. In den vergangenen Jahren ist insbesondere die generative Modellierung in ein neues Zeitalter eingetreten, wie durch die Popularität von Sprachmodellen wie ChatGPT oder Bildgeneratoren wie DALL-E und Stable Diffusion gezeigt wird. Diese Modelle sind in der Lage, menschenähnliche Texte zu schreiben, fotorealistische Bilder zu erzeugen und sogar Videos und Musik zu kreieren, wodurch sie die Art und Weise verändern, wie KI mit kreativen Prozessen interagiert.

Auch für industrielle Anwendungen hat diese Entwicklung eine immense Bedeutung, insbesondere im Bereich des Maschinellen Sehens, bei dem es darum geht, visuelle Informationen zu verarbeiten und zu interpretieren. Es ist die Grundlage vieler automatisierter Prozesse, wie etwa der Qualitätskontrolle in Produktionslinien, der Objekterkennung in der Robotik oder der Bildklassifikation in der Medizin. Ein zentrales Ziel dabei ist es, Modelle zu entwickeln, die in der Lage sind, zuverlässig Muster in visuellen Daten zu erkennen und daraus Entscheidungen zu treffen.

Ein wesentlicher Faktor für den Erfolg dieser Modelle ist die Verfügbarkeit von Daten. Industrielle Anwendungsfälle sind allerdings von domänenspezifischen Anforderungen und einer hohen Komplexität geprägt, weshalb herkömmliche Ansätze der Datenbeschaffung oft an ihre Grenzen stoßen \parencite{Kraljevski2023smalldata,Lu2024syntheticdatareview}. Die manuelle Erfassung und Annotation von Daten ist zeitaufwendig, teuer und in vielen Fällen nicht praktikabel. Zudem sind die verfügbaren Datenmengen in hochspezialisierten Anwendungen oft zu gering oder zu homogen, um ein Modell mit der nötigen Robustheit und Generalisierungsfähigkeit zu trainieren. Dies führt dazu, dass die Modelle im Umgang mit neuen Daten nicht die gewünschte Leistung erzielen, da sie mit Out-of-Distribution (OOD)-Daten konfrontiert werden, also mit Daten, welche nicht aus der gleichen Verteilung stammen wie die Trainingsdaten.

Hier kommen synthetische Daten ins Spiel, die durch generative Modelle künstlich erzeugt werden. Der Vorteil dieser synthetischen Daten liegt darin, dass sie in beliebiger Menge und Varianz produziert werden können, um die Datenvielfalt zu erhöhen und das Modell auf verschiedenste Szenarien vorzubereiten. Trotz dieser Vorteile ist auch die synthetische Datengenerierung von den Herausforderungen der industriellen Anwendungen betroffen und scheitert oft bei der realistischen Darstellung komplexer, domänenspezifischer Konzepte aus einer limitierten Menge an Beispielen \parencite{Lu2024syntheticdatareview}.

Die vorliegende Arbeit basiert auf einem Pflichtpraktikum am Fraunhofer Institut für Produktionsanlagen und Konstruktionstechnik (IPK) in Berlin, wo diese Herausforderungen im Rahmen eines konkreten Anwendungsfalls erlebt wurden. Es handelte sich dabei um ein Forschungsprojekt zur Entwicklung eines KI-Systems für die Klassifikation von Gebrauchsgegenständen in der Recyclingwirtschaft. Dafür sollten synthetische Daten der teils sehr komplexen Objekte generiert werden, um sie in unterschiedlichen Gebrauchszuständen darzustellen. Obwohl verschiedene Ansätze getestet wurden, erzielten sie nur mittelmäßige Ergebnisse, weshalb die Suche nach einer effektiven Methode zur Generierung synthetischer Daten für industrielle Anwendungen auch für diese Arbeit die zentrale Motivation darstellt.

\section{Zielsetzung} \label{sec:goal}

Vor diesem Hintergrund befasst sich die Arbeit mit der Untersuchung der Methode DA-Fusion \parencite{Trabucco2023dafusion}, welche ein vortrainiertes Stable Diffusion-Modell benutzt, um automatisch semantisch sinnvolle Variationen von Bildern zu generieren, ohne dass dem Modell die dargestellten Konzepte vorher bekannt sein müssen.

% Darüber hinaus wird ein besonderer Fokus auf die Integration der synthetischen Daten in das Contrastive Learning gelegt
Ein weiterer wichtiger Teil der Arbeit ist der Einsatz von Contrastive Learning, einer Methode, die in den letzten Jahren immer mehr an Bedeutung gewonnen hat. Contrastive Learning zielt darauf ab, repräsentative Merkmale von Daten zu lernen, indem es ähnliche Datenpunkte in einem latenten Repräsentationsraum näher zusammenbringt und unähnliche weiter voneinander entfernt. Diese Technik erwies sich als besonders nützlich, um robuste und generalisierbare Repräsentationen zu erlernen \parencite{Liu2021understandimprovecontrastivelearning}.

Mit der Kombination von DA-Fusion und Contrastive Learning soll dabei nicht nur die Generalisierungsfähigkeit der Bildklassifikation verbessert werden, sondern auch untersucht werden, ob mangelhafte synthetische Daten einen Mehrwert für die Modellleistung haben können, wenn sie ausschließlich als negative Beispiele im Contrastive Learning integriert werden. Hier bietet DA-Fusion die Möglichkeit, einerseits "normale" synthetische Daten zu generieren und andererseits "Near Out-of-Distribution" (Near OOD)-Daten.

\newpage

Konkret sollen somit zwei zentrale Forschungsfragen beantwortet werden:

\begin{enumerate}
    \item Kann DA-Fusion synthetische Daten für die Recyclingwirtschaft erzeugen, die die Generalisierungsfähigkeit des Modells im Supervised Contrastive Learning verbessern?
    \item Können Near OOD-Augmentationen im Supervised Contrastive Learning dazu beitragen, die Robustheit des Modells gegenüber OOD-Daten zu erhöhen und die Repräsentationen von In-Distribution-Daten weiter zu verbessern?
\end{enumerate}

\section{Aufbau der Arbeit} \label{sec:structure}

Die vorliegende Arbeit gliedert sich in sechs Kapitel. 

In \autoref{ch:theory} werden zunächst die theoretischen Grundlagen gelegt, die das Verständnis für die darauffolgenden Kapitel ermöglichen. Es beginnt mit einer Einführung in das Maschinelle Lernen, wobei überwachte und unüberwachte Lernverfahren erklärt werden, bevor tiefer auf Deep Learning und neuronale Netze eingegangen wird. Ein besonderes Augenmerk wird dabei auf die Bedeutung der Datenaugmentation und das Konzept der Out-of-Distribution-Daten gelegt. Im Anschluss daran wird auf die verschiedenen Ansätze zur Generierung synthetischer Daten eingegangen, darunter auch die Diffusionsmodelle, auf denen DA-Fusion aufbaut. Das Kapitel schließt mit einer ausführlichen Betrachtung von Contrastive Learning, sowie der Beschreibung des Anwendungsfalls zur Klassifikation von Gebrauchsgegenständen in der Recyclingwirtschaft. Hier werden die Herausforderungen bei der Generierung synthetischer Daten, die Möglichkeit zur Nutzung synthetischer Daten als negativ-Beispiele im Contrastive Learning und der eigene Ansatz zur Integration von DA-Fusion in das Supervised Contrastive Learning beschrieben.

\autoref{ch:methodology} beschreibt das methodische Vorgehen. Hier wird zunächst der verwendete MVIP-Datensatz detailliert erläutert, der verwendete Teildatensatz definiert und die notwendigen Vorverarbeitungsschritte erklärt. Danach wird auf die Implementierung der DA-Fusion-Methode und des Supervised Contrastive Learning eingegangen. Schließlich wird der Versuchsaufbau beschrieben, der sowohl die Generierung der synthetischen Daten mittels DA-Fusion als auch die Trainings- Testdurchläufe umfasst.

In \autoref{ch:results} werden die Ergebnisse der Arbeit präsentiert. Zunächst werden die erzeugten synthetischen Daten vorgestellt, aufgeteilt in die Kategorien In-Distribution- und Near-Out-of-Distribution-Daten. Daraufhin folgen die Ergebnisse der Trainings- und Testdurchläufe aus dem Supervised Contrastive Learning, die das Pre-Training der Repräsentationen und die darauf aufbauende lineare Klassifikation betreffen. Im Anschluss werden die erzielten Resultate mit und ohne In-Distribution-Augmentationen sowie mit und ohne Near Out-of-Distribution-Augmentationen miteinander verglichen.

\autoref{ch:discussion} widmet sich der Diskussion der Ergebnisse. Hier wird insbesondere die Eignung von DA-Fusion zur Generierung synthetischer Daten bewertet. Zudem wird die Wirksamkeit der Near Out-of-Distribution-Daten im Rahmen des Supervised Contrastive Learning diskutiert, um zu analysieren, ob diese zur Verbesserung der Modellleistung beitragen.

Abschließend folgt in \autoref{ch:conclusion} eine Zusammenfassung der wichtigsten Erkenntnisse der Arbeit. Hier werden die Forschungsfragen beantwortet und ein Ausblick auf mögliche Weiterentwicklungen und zukünftige Forschungsarbeiten gegeben.