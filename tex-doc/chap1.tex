% !TEX root = Bachelorarbeit Synthetische Daten.tex
\chapter{Einleitung} \label{ch:intro}

Seit der Einführung von Convolutional Neural Networks (CNNs) hat sich die Bildklassifikation erheblich weiterentwickelt und ist zu einem zentralen Anwendungsgebiet des maschinellen Lernens geworden. Die Leistungsfähigkeit dieser Modelle hängt jedoch stark von der Qualität und Vielfalt der Trainingsdaten ab. Um die Generalisierungsfähigkeit und Robustheit von Klassifikationsmodellen zu verbessern, ist es gängige Praxis, die Menge und Vielfalt der Trainingsdaten durch Datenaugmentation zu erhöhen. Dabei werden die Bilder durch Transformationen wie Rotation, Skalierung oder Helligkeitsanpassung verändert, um die Modellgenauigkeit zu steigern.

\section{Motivation} \label{sec:motivation}

In der heutigen Zeit werden maschinelle Lernmodelle zunehmend in anspruchsvollen Anwendungen wie der Bildklassifikation eingesetzt. Diese Modelle sind jedoch oft anfällig für eine unzureichende Generalisierungsfähigkeit, insbesondere wenn sie auf Daten stoßen, die außerhalb der Verteilung des Trainingsdatensatzes liegen, sogenannte Out-of-Distribution (OOD) Daten. Um die Generalisierung zu verbessern, ist es gängige Praxis, die Menge und Vielfalt der Trainingsdaten durch Datenaugmentation zu erhöhen. Dabei stoßen konventionelle Methoden jedoch an ihre Grenzen, wenn es darum geht, reale, komplexe Variationen oder Gebrauchsspuren in den Bildern zu simulieren.

Die jüngsten Fortschritte in generativen Modellen, wie Variational Autoencoders (VAE), Generative Adversarial Networks (GANs) und Diffusion Models, haben das Potenzial, diese Herausforderung durch die Erzeugung synthetischer Daten zu bewältigen. Speziell die Kombination von Diffusion Models mit kontrastivem Lernen bietet vielversprechende Ansätze, um die Robustheit und Genauigkeit von Klassifikationsmodellen zu verbessern. Diese Arbeit zielt darauf ab, das Potenzial der Stable Diffusion-basierten Datenaugmentation zu untersuchen und deren Effektivität im Supervised Contrastive Learning zu bewerten.

\section{Zielsetzung} \label{sec:goal}

Das Hauptziel dieser Arbeit besteht darin, die Eignung der Stable Diffusion-basierten Augmentationsmethode DA-Fusion zur Generierung synthetischer Daten zu untersuchen und deren Einfluss auf die Bildklassifikation zu bewerten. Ein besonderer Fokus liegt hierbei auf der Rolle von Near Out-of-Distribution-Augmentationen im Rahmen des Supervised Contrastive Learning. Es wird analysiert, wie sich synthetische Daten auf die Generalisierungsfähigkeit und Robustheit der Modelle gegenüber unbekannten Datenverteilungen auswirken.

Die zentralen Forschungsfragen dieser Arbeit lauten:
\begin{itemize}
    \item Wie gut eignet sich DA-Fusion zur Generierung synthetischer Daten für die Bildklassifikation? \item Inwiefern tragen Near Out-of-Distribution-Augmentationen im Supervised Contrastive Learning zur Verbesserung der Modellgenauigkeit bei?
\end{itemize}

% Forschungsfragen und Hypothesen
% Im vorherigen Kapitel wurden Forschungslücken identifiziert, die sich aus der Verwendung von DA-Fusion im Supervised Contrastive Learning und der Verwendung von Near OOD-Augmentationen für das Negative Sampling ergeben. Um diese Lücken zu schließen, werden die folgenden Forschungsfragen und Hypothesen formuliert:
    % \textbf{Forschungsfrage 1:} Kann DA-Fusion für den EIBA-Datensatz synthetische Augmentationen erzeugen, die die Generalisierungsfähigkeit im Supervised Contrastive Learning verbessern?
        % Durch Beantwortung dieser Frage soll festgestellt werden, ob sich DA-Fusion grundsätzlich eignet, um die Herausforderungen der synthetischen Datengenerierung in Anwendungsfällen wie dem EIBA-Datensatz zu bewältigen (genaueres zum Datensatz in Abschnitt \ref{sec:dataset}). Dazu wird DA-Fusion auf "normale" Weise verwendet, d.h. es werden synthetische In-Distribution Daten generiert, die die Repräsentationen der realen Daten verbessern sollen. Es wird untersucht, ob die Verwendung der Augmentationen im Supervised Contrastive Learning dazu beiträgt, die Leistung des Modells für zuvor ungesehene Daten zu verbessern.
    % \textbf{Forschungsfrage 2:} Trägt die Verwendung von Out-of-Distribution (OOD) Augmentationen im Supervised Contrastive Learning dazu bei, die Robustheit des Modells gegenüber OOD-Daten zu erhöhen und die Repräsentationen von In-Distribution-Daten zu verbessern?
        % Im Rahmen dieser Frage wird untersucht, ob Near OOD-Augmentationen \textemdash also synthetische Daten, welche aus den echten Objekten abgeleitet sind, diese aber nicht akkurat darstellen müssen \textemdash im Supervised Contrastive Learning einen Mehrwert bieten, indem sie die Repräsentationen der In-Distribution-Daten verbessern und die Robustheit gegenüber OOD-Daten erhöhen. Dazu wird eine neue Negative Sampling-Strategie für das Supervised Contrastive Learning verwendet, die es ermöglicht, für jeden Anchor genau die Near OOD-Augmentationen als negativ-Beispiele heranzuziehen, die aus einem Beispiel der Anchor-Klasse generiert wurden. Es wird untersucht, ob so die Generalisierungsfähigkeit und die Robustheit gegenüber OOD-Daten noch weiter gesteigert werden kann.

\section{Aufbau der Arbeit} \label{sec:structure}

Diese Arbeit gliedert sich wie folgt:

\begin{itemize}
    \item \textbf{Theoretische Grundlagen} – In diesem Kapitel werden die wichtigsten Konzepte und Methoden der Datenaugmentation, generativen Modelle und des Contrastive Learning erläutert, die für das Verständnis der Arbeit notwendig sind.
    \item \textbf{Methodisches Vorgehen} – Dieses Kapitel beschreibt die verwendeten Datensätze, den experimentellen Aufbau und die Evaluationsmethoden. Außerdem wird die Implementierung der Stable Diffusion-basierten Augmentation und des Supervised Contrastive Learning erklärt.
    \item \textbf{Ergebnisse} – Die experimentellen Ergebnisse, die die Performance der synthetischen Daten und die Wirksamkeit der Augmentationen im Vergleich zu herkömmlichen Methoden aufzeigen, werden hier präsentiert.
    \item \textbf{Diskussion} – In der Diskussion werden die Ergebnisse interpretiert und im Kontext der Forschungsfragen bewertet.
    \item \textbf{Fazit} – Abschließend werden die wichtigsten Erkenntnisse zusammengefasst, die Forschungsfragen beantwortet und ein Ausblick auf potenzielle Weiterentwicklungen gegeben.
\end{itemize}