% !TEX root = Bachelorarbeit Synthetische Daten.tex
\chapter{Einleitung} \label{ch:intro}

Es folgt zunächst eine Einleitung in das Thema der Arbeit, die Motivation für die Untersuchung, die Zielsetzung und der Aufbau der Arbeit.

\section{Motivation} \label{sec:motivation}

Die rasanten Enwicklungen in den Bereichen der Künstlichen Intelligenz (KI) und des Maschinellen Lernens (ML) decken eine breite Palette von Anwendungen ab, die an immer mehr Stellen in unserem Alltag Einzug halten. In den vergangenen Jahren ist insbesondere die generative Modellierung in ein neues Zeitalter eingetreten, wie durch die Popularität von Sprachmodellen wie ChatGPT oder Bildgeneratoren wie DALL-E und Stable Diffusion gezeigt wird. Diese generativen Modelle sind in der Lage, menschenähnliche Texte zu schreiben, fotorealistische Bilder zu erzeugen und sogar Videos und Musik zu kreieren, wodurch sie die Art und Weise verändern, wie KI mit kreativen Prozessen interagiert.

% Auch für industrielle Anwendungen ist diese Entwicklung von großer Bedeutung, insbesondere im Bereich des Maschinellen Sehens, bei dem es darum geht, visuelle Informationen zu verarbeiten und zu interpretieren. Es ist die Grundlage vieler automatisierter Prozesse, wie der Qualitätskontrolle in der Produktion. Hier kommen generative Modelle vor allem zur Erzeugung von synthetischen Daten zum Einsatz. Diese Daten können die Menge und Vielfalt der Trainingsdaten erhöhen und so die Generalisierungsfähigkeit von Modellen verbessern. ...
Auch für industrielle Anwendungen hat diese Entwicklung eine immense Bedeutung, insbesondere im Bereich des Maschinellen Sehens, bei dem es darum geht, visuelle Informationen zu verarbeiten und zu interpretieren. Es ist die Grundlage vieler automatisierter Prozesse, wie etwa der Qualitätskontrolle in Produktionslinien, der Objekterkennung in autonomen Fahrzeugen oder der Fehlererkennung in medizinischen Bildern. Ein zentrales Ziel dabei ist es, Modelle zu entwickeln, die in der Lage sind, zuverlässig Muster in visuellen Daten zu erkennen und daraus Entscheidungen zu treffen.

% Industrielle Anwendungsfälle sind allerdings von einer hohen Komplexität und domänenspezifischen Anforderungen geprägt, ...
Ein wesentlicher Faktor für den Erfolg dieser Modelle ist die Verfügbarkeit von Daten. Industrielle Anwendungsfälle sind allerdings von domänenspezifischen Anforderungen und einer hohen Komplexität geprägt, weshalb herkömmliche Ansätze der Datenbeschaffung oft an ihre Grenzen stoßen. Die manuelle Erfassung und Annotation von Daten ist zeitaufwendig, teuer und in vielen Fällen nicht praktikabel. Zudem sind die verfügbaren Datenmengen in hochspezialisierten Anwendungen oft zu gering oder zu homogen, um ein Modell mit der nötigen Robustheit und Generalisierungsfähigkeit zu trainieren. Dies führt dazu, dass die Modelle häufig in realen Umgebungen nicht die gewünschte Leistung erzielen, da sie mit Out-of-Distribution (OOD)-Daten konfrontiert werden, also mit Daten, welche nicht aus der gleichen Verteilung stammen wie die Trainingsdaten.

Hier kommen synthetische Daten ins Spiel, die durch generative Modelle künstlich erzeugt werden. Der Vorteil dieser synthetischen Daten liegt darin, dass sie in beliebiger Menge und Varianz produziert werden können, um die Datenvielfalt zu erhöhen und das Modell auf verschiedenste Szenarien vorzubereiten. Trotz dieser Vorteile ist auch die synthetische Datengenerierung von den Herausforderungen der industriellen Anwendungen betroffen und scheitert oft bei der realistischen Darstellung komplexer, domänenspezifischer Konzepte aus einer limitierten Menge an Beispielen \parencite{Lu2024syntheticdatareview}.

Vor diesem Hintergrund befasst sich diese Arbeit mit der Untersuchung der auf Stable Diffusion basierenden Methode DA-Fusion \parencite{Trabucco2023dafusion}, welche eine effektive Generierung semantischer Augmentationen verspricht, ohne dass das vortrainierte Stable Diffusion-Modell Vorwissen zu den gegebenen Konzepten haben muss. Dies ist besonders relevant, um die Herausforderungen der Variabilität der Domänenspezifik zu bewältigen.

Darüber hinaus wird in dieser Arbeit der Fokus auf die Integration dieser synthetischen Daten in Contrastive Learning gelegt, einer Methode, die in den letzten Jahren immer mehr an Bedeutung gewonnen hat. Contrastive Learning zielt darauf ab, repräsentative Merkmale von Daten zu lernen, indem es ähnliche Datenpunkte in einem hochdimensionalen Merkmalsraum näher zusammenbringt und unähnliche weiter voneinander entfernt. Diese Technik erweist sich als besonders nützlich, um robuste und generalisierbare Repräsentationen zu erlernen.

% Praktikum?

\section{Zielsetzung} \label{sec:goal}

Das Hauptziel dieser Arbeit besteht darin, die Eignung der Stable Diffusion-basierten Augmentationsmethode DA-Fusion zur Generierung synthetischer Daten für einen spezifischen Anwendungsfall in der Recyclingwirtschaft zu untersuchen und deren Einfluss auf die Bildklassifikation zu bewerten. Die Analyse fokussiert sich auf zwei zentrale Aspekte: Erstens wird untersucht, ob DA-Fusion in der Lage ist, hochwertige synthetische Augmentationen zu erzeugen, die zur Verbesserung der Generalisierungsfähigkeit von Bildklassifikationsmodellen beitragen können. Zweitens steht die Integration von Near Out-of-Distribution (Near OOD)-Augmentationen in das Supervised Contrastive Learning (SCL) im Mittelpunkt, um zu evaluieren, welchen Einfluss diese speziellen Augmentationen auf die Robustheit der Modellleistung haben.

In industriellen Anwendungsbereichen wie der Recyclingwirtschaft ist die Verfügbarkeit großer und vielfältiger Datensätze oft limitiert. Daher stellt die Erzeugung synthetischer Daten durch DA-Fusion eine vielversprechende Methode dar, um die Datenbasis zu erweitern und die Leistungsfähigkeit von Bildklassifikationsmodellen zu verbessern. Ein wesentlicher Aspekt ist dabei die Frage, ob durch diese generierten Daten eine höhere Generalisierungsfähigkeit erreicht werden kann, d.h., ob Modelle, die mit augmentierten Datensätzen trainiert werden, besser auf bisher ungesehene Daten reagieren können.

Neben den Standard-Augmentationen soll auch untersucht werden, inwiefern Near OOD-Augmentationen, also Daten, die sich leicht von den In-Distribution-Daten unterscheiden, aber nicht vollständig aus einer fremden Domäne stammen, einen Mehrwert für das Supervised Contrastive Learning bieten. Contrastive Learning ist darauf ausgerichtet, robuste Repräsentationen zu lernen, indem ähnliche Daten näher zusammen und unterschiedliche Daten weiter voneinander entfernt im latenten Raum positioniert werden. Hierbei stellt sich die Frage, ob Near OOD-Daten als sogenannte *Hard Negatives* \textemdash also besonders herausfordernde Beispiele \textemdash die Fähigkeit des Modells verbessern können, zwischen In-Distribution- und OOD-Daten zu unterscheiden und somit die Robustheit gegenüber unbekannten Datenverteilungen zu erhöhen.

Die zentralen Forschungsfragen, die in dieser Arbeit adressiert werden, lauten daher:

\begin{itemize}
    \item Kann DA-Fusion synthetische Augmentationen für den gegebenen Anwendungsfall in der Recyclingwirtschaft erzeugen, die die Generalisierungsfähigkeit im Supervised Contrastive Learning verbessern
    \item Trägt die Verwendung von Near OOD-Augmentationen im Supervised Contrastive Learning dazu bei, die Robustheit des Modells gegenüber OOD-Daten zu erhöhen und die Repräsentationen von In-Distribution-Daten weiter zu verbessern?
\end{itemize}

Die Beantwortung dieser Fragen soll nicht nur zur Verbesserung der Klassifikationsmodelle im spezifischen Kontext der Recyclingwirtschaft beitragen, sondern auch generell Aufschluss darüber geben, wie synthetische Daten und OOD-Augmentationen die Modellleistung in der Bildverarbeitung beeinflussen können.

\section{Aufbau der Arbeit} \label{sec:structure}

Die vorliegende Arbeit gliedert sich in sechs Kapitel. 

In \autoref{ch:theory} werden zunächst die theoretischen Grundlagen gelegt, die das Verständnis für die darauffolgenden Kapitel ermöglichen. Es beginnt mit einer Einführung in das Maschinelle Lernen, wobei sowohl überwachte als auch unüberwachte Lernverfahren erklärt werden, bevor tiefer auf Deep Learning und neuronale Netze eingegangen wird. Ein besonderes Augenmerk wird dabei auf die Bedeutung der Datenaugmentation und das Konzept der Out-of-Distribution-Daten gelegt. Im Anschluss daran wird auf die verschiedenen Ansätze zur Generierung synthetischer Daten eingegangen, darunter Variational Autoencoder (VAE), Generative Adversarial Networks (GANs) und Diffusionsmodelle, wobei der Fokus auf Stable Diffusion und DA-Fusion liegt. Das Kapitel schließt mit einer ausführlichen Betrachtung von Contrastive Learning, insbesondere des Supervised Contrastive Learning (SCL), sowie der Beschreibung des Anwendungsfalls zur Klassifikation von Gebrauchsgegenständen in der Recyclingwirtschaft. Hier werden die Herausforderungen bei der Generierung synthetischer Daten, die Möglichkeit zur Nutzung synthetischer Daten als negative Beispiele im SCL und der eigene Ansatz zur Integration von DA-Fusion in das SCL beschrieben.

\autoref{ch:methodology} beschreibt das methodische Vorgehen. Hier wird zunächst der verwendete MVIP-Datensatz detailliert erläutert, der verwendete Teildatensatz definiert und die notwendigen Vorverarbeitungsschritte erklärt. Danach wird auf die Implementierung der DA-Fusion-Methode und des Supervised Contrastive Learning eingegangen. Das Kapitel schließt mit einer Beschreibung des Versuchsaufbaus, der sowohl die Generierung der synthetischen Daten mittels DA-Fusion als auch die Trainingsdurchläufe und die verwendeten Evaluationsmethoden umfasst.

In \autoref{ch:results} werden die Ergebnisse der Arbeit präsentiert. Zunächst werden die erzeugten synthetischen Daten vorgestellt und in zwei Kategorien eingeteilt: In-Distribution- und Near-Out-of-Distribution-Daten. Daraufhin folgen die Ergebnisse der Trainings- und Testdurchläufe, die das Pre-Training mittels Supervised Contrastive Learning und die darauf aufbauende lineare Klassifikation betreffen. Im Anschluss werden die erzielten Resultate mit und ohne In-Distribution-Augmentationen sowie mit und ohne Near OOD-Augmentationen miteinander verglichen.

\autoref{ch:discussion} widmet sich der Diskussion der Ergebnisse. Hier wird insbesondere die Eignung von DA-Fusion zur Generierung synthetischer Daten bewertet. Zudem wird die Wirksamkeit der Near Out-of-Distribution-Daten im Rahmen des Supervised Contrastive Learning diskutiert, um zu analysieren, ob diese zur Verbesserung der Modellleistung beitragen.

Abschließend folgt in \autoref{ch:conclusion} eine Zusammenfassung der wichtigsten Erkenntnisse der Arbeit. Hier werden die Forschungsfragen beantwortet und ein Ausblick auf mögliche Weiterentwicklungen und zukünftige Forschungsarbeiten gegeben.