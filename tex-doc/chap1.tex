% !TEX root = Bachelorarbeit Synthetische Daten.tex
\chapter{Einleitung} \label{ch:intro}

...

\section{Motivation} \label{sec:motivation}

...
% Praktikum

\section{Zielsetzung} \label{sec:goal}

...

% Forschungsfragen und Hypothesen

%Im vorherigen Kapitel wurden Forschungslücken identifiziert, die sich aus der Verwendung von DA-Fusion im Supervised Contrastive Learning und der Verwendung von Near OOD-Augmentationen für das Negative Sampling ergeben. Um diese Lücken zu schließen, werden die folgenden Forschungsfragen und Hypothesen formuliert:

\textbf{Forschungsfrage 1:} Kann DA-Fusion für den EIBA-Datensatz synthetische Augmentationen erzeugen, die die Generalisierungsfähigkeit im Supervised Contrastive Learning verbessern?

Durch Beantwortung dieser Frage soll festgestellt werden, ob sich DA-Fusion grundsätzlich eignet, um die Herausforderungen der synthetischen Datengenerierung in Anwendungsfällen wie dem EIBA-Datensatz zu bewältigen (genaueres zum Datensatz in Abschnitt \ref{sec:dataset}). Dazu wird DA-Fusion auf "normale" Weise verwendet, d.h. es werden synthetische In-Distribution Daten generiert, die die Repräsentationen der realen Daten verbessern sollen. Es wird untersucht, ob die Verwendung der Augmentationen im Supervised Contrastive Learning dazu beiträgt, die Leistung des Modells für zuvor ungesehene Daten zu verbessern.

% Hypothese
...

\textbf{Forschungsfrage 2:} Trägt die Verwendung von Out-of-Distribution (OOD) Augmentationen im Supervised Contrastive Learning dazu bei, die Robustheit des Modells gegenüber OOD-Daten zu erhöhen und die Repräsentationen von In-Distribution-Daten zu verbessern?

Im Rahmen dieser Frage wird untersucht, ob Near OOD-Augmentationen \textemdash also synthetische Daten, welche aus den echten Objekten abgeleitet sind, diese aber nicht akkurat darstellen müssen \textemdash im Supervised Contrastive Learning einen Mehrwert bieten, indem sie die Repräsentationen der In-Distribution-Daten verbessern und die Robustheit gegenüber OOD-Daten erhöhen. Dazu wird eine neue Negative Sampling-Strategie für das Supervised Contrastive Learning verwendet, die es ermöglicht, für jeden Anchor genau die Near OOD-Augmentationen als negativ-Beispiele heranzuziehen, die aus einem Beispiel der Anchor-Klasse generiert wurden. Es wird untersucht, ob so die Generalisierungsfähigkeit und die Robustheit gegenüber OOD-Daten noch weiter gesteigert werden kann.

% Hypothese
...

\section{Aufbau der Arbeit} \label{sec:structure}

...
