% !TEX root = Bachelorarbeit Synthetische Daten.tex

% Für diese Vorlage wurde die Klasse "scrreprt" von KOMA-Script gewählt, die in etwa der Standardklasse "report" entspricht, allerdings wesentlich mehr Möglichkeiten bietet und im gewissen "moderner" ist.
% Ausführliche Dokumentation: http://mirrors.ctan.org/macros/latex/contrib/koma-script/doc/scrguide.pdf
\documentclass[
  fontsize=12pt,
  paper=A4,
  bibliography=totoc, % Literaturverzeichnis ins Inhaltsverzeichnis
  listof=totoc, % Andere Verzeichnisse ebenfalls ins Inhaltsverzeichnis
  %fleqn, % Abgesetzte Formeln linksbündig
  DIV=12, % Für die Satzspiegelkonstruktion - siehe KOMA-Doku
  BCOR=1mm, % Bindekorrektur (linker Rand) - evtl. anpassen
  english,ngerman, % Die im Text verwendeten Sprachen (u.a. für das Paket babel); die letztgenannte (!) Sprache ist die Standardsprache; "n"german steht für die neue Rechtschreibung
  usegeometry, % Weil (s.u.) das Paket geometr verwendet wird
  parskip=half- % Absätze ohne Einzug, halbe Zeile Abstand
]{scrreprt}

\usepackage{caption}
\captionsetup[figure]{justification=centering, format=plain, list=yes}
\captionsetup[table]{format=plain, list=yes}
% Beschriftungen für Tabellen kommen linksbündig über die Tabelle
\KOMAoption{captions}{tableheading,nooneline}
%\setcaptionalignment[figure]{c}
%\setcaptionalignment[table]{l}

% Für die Titelseite benötigt
\usepackage{geometry}

% Standardpaket für Lokalisation, siehe Option "ngerman" oben
\usepackage{babel}
% Optimierten Trennmuster
\babelprovide[hyphenrules=ngerman-x-latest]{ngerman}

% Standardpaket für mathematische Zusatzfunktionen
\usepackage{amsmath}

% Hauptschrift Libertinus
\usepackage{libertinus-otf}
% "Schreibmaschinenschrift" Anonymous Pro, angepasst
\usepackage{AnonymousPro}
\setmonofont{AnonymousPro}[Scale=MatchLowercase,FakeStretch=0.85]

% Etwas größerer Zeilenabstand als im Buchsatz
\linespread{1.1}

% Paket für Feinkorrekturen an der Typographie für ausgewogeneres Schriftbild
\usepackage{microtype}

% Paket für kontextsensitive Anführungszeichen
\usepackage{csquotes}
% Shortcut, damit aus " richtige Anführungszeichen werden, je nach Sprache
\MakeOuterQuote{"}

% Paket, das den Befehl \includegraphics ermöglicht
\usepackage{graphicx}

% Komfortablere Aufzählungen als in Standard-LaTeX
\usepackage{enumitem}

% Paket für eigene Farbdefinitionen
\usepackage[dvipsnames]{xcolor}
% "Hausfarben" der HAW
\definecolor{haw}{HTML}{003CA0}
\definecolor{haw2}{HTML}{0096D2}
\definecolor{haw3}{HTML}{A0BEDC}
% Farben für die Versuchsreihen
\definecolor{exp1}{HTML}{7D54B2}
\definecolor{exp2}{HTML}{479A5F}
\definecolor{exp3}{HTML}{EDB732}

% Typographisch anspruchsvolle Tabellen
\usepackage{booktabs}

% Erstellen des Literaturverzeichnisses; Stil APA eingestellt
\usepackage[style=apa]{biblatex}
% Bibtex-Datei mit den Literaturangaben
\addbibresource{library.bib}

% Erzeugung von Grafiken mit PGF/TikZ
%\usepackage{tikz}
%\usetikzlibrary{calc,intersections,angles,3d}

% Codeblöcke
\usepackage{listings}
% Anpassung des Erscheinungsbildes des Codeblocks
\lstdefinestyle{mystyle}{
    backgroundcolor=\color{gray!20},
    keywordstyle=\color{haw2},
    numberstyle=\footnotesize\color{haw},
    basicstyle=\ttfamily\small,
    captionpos=t,
    frame=single,
    framerule=0pt,
    keepspaces=true,
    numbers=left,
    numbersep=6pt,
    belowcaptionskip=1em,
    aboveskip=\bigskipamount,
}
\lstset{style=mystyle}
% Damit es "Codeblock" und nicht "Listing" heißt
\renewcommand{\lstlistingname}{Codeblock}

% Verlinkung innerhalb des PDF-Dokuments, für PDF-Lesezeichen und PDF-Metadaten; dieses Paket sollte üblicherweise immer als letztes geladen werden
\usepackage[colorlinks=true,allcolors=haw,hyperfootnotes=false,pageanchor=true,linktoc=all]{hyperref}

% Für Druckversion obige Zeile durch folgende ersetzen, damit Links nicht blau dargestellt werden:
% \usepackage[draft]{hyperref}

% Metadaten des PDF-Dokumentes
\hypersetup{pdfauthor={Paul Hofmann}}
\hypersetup{pdftitle={Contrastive Learning mit Stable Diffusion-basierter Datenaugmentation}}