% !TEX root = Bachelorarbeit Synthetische Daten.tex
\chapter{Fazit} \label{ch:conclusion}

Eine Vielzahl von Entwicklungen im Bereich des Maschinellen Lernens, vor allem in Bezug auf generative Modelle, haben zu neuen Möglichkeiten in der synthetischen Datengenerierung geführt. Diese Arbeit untersuchte die Eignung von Diffusionsmodellen, insbesondere DA-Fusion, in Kombination mit Supervised Contrastive Learning zur Verbesserung der Bildklassifikation. Ziel war es, einen Ansatz zu finden, der die Herausforderung der synthetischen Datengenerierung für komplexe Objekte mit feinen Unterschieden adressiert.

\section{Zusammenfassung der wichtigsten Erkenntnisse} \label{sec:summary}

Die durchgeführten Experimente zeigten, dass DA-Fusion bei der Generierung von ID-Augmentationen deutliche Verbesserungen der Modellleistung erzielte. Durch die Erzeugung synthetischer Gebrauchszustände und die Erhöhung der Variabilität innerhalb der Daten konnte die Top-1-Accuracy um etwa 6 Prozentpunkte gesteigert werden, was auf eine bessere Generalisierungsfähigkeit des Modells hindeutet.

Allerdings traten Schwierigkeiten auf, als Near OOD-Augmentationen als negativ-Beispiele genutzt wurden. Statt der erwarteten Leistungssteigerung verschlechterte sich die Top-1-Accuracy, was auf mehrere Faktoren zurückzuführen ist, darunter die Qualität der generierten OOD-Daten und die Komplexität der Implementierung der Sampling-Strategie im Supervised Contrasitve Learning. Viele der synthetischen Near OOD-Daten wiesen Artefakte auf oder entfernten sich zu weit von den ID-Daten, sodass sie keine geeigneten negativ-Beispiele mehr darstellten.

\section{Beantwortung der Forschungsfragen} \label{sec:research-questions-answers}

Die erste Forschungsfrage befasste sich mit der Frage, ob DA-Fusion zur Generierung synthetischer Daten geeignet ist und wie sich diese Daten auf die Generalisierungsfähigkeit des Modells auswirken. Die Experimente zeigten, dass DA-Fusion erfolgreich ID-Augmentationen generieren kann, die die Generalisierungsfähigkeit verbessern. Diese Augmentationen ermöglichten eine größere Variabilität innerhalb der Daten und stärkten die Robustheit des Modells gegenüber feinen Unterschieden in den Klassen.

Die zweite Forschungsfrage beschäftigte sich damit, ob Near OOD-Augmentationen im Supervised Contrastive Learning als Hard Negatives dienen können. Hier konnte die erwartete Leistungssteigerung nicht bestätigt werden. Stattdessen führte der Einsatz von Near OOD-Daten zu einer Verschlechterung der Modellleistung. Dies lag vermutlich an der geringen Qualität der generierten OOD-Daten, die oft zu weit von den ID-Daten entfernt waren. Sie fungierten nicht mehr als semantisch nahe negative Beispiele, was den Lernprozess im SCL negativ beeinflusste.

\section{Ausblick und potenzielle Weiterentwicklungen} \label{sec:outlook}

Die Ergebnisse dieser Arbeit eröffnen mehrere potenzielle Ansätze für zukünftige Forschung. Ein Schwerpunkt sollte auf der Verbesserung der generierten OOD-Daten liegen, um nicht zu weit von den ID-Daten abzuweichen und tatsächlich herausfordernde negativ-Beispiele zu bieten. Hier könnten alternative Techniken zum Fine-tuning des in DA-Fusion verwendeten Stable Diffusion-Modells verwendet werden.

Darüber hinaus könnte die Implementierung von Near OOD-Daten als Hard Negatives im Supervised Contrastive Learning weiter optimiert werden. Dabei könnten fortgeschrittenere Sampling-Strategien entwickelt werden.

Auch die Forschung zur Frage, wie Modelle aus minderwertigen synthetischen Daten lernen können, könnte weiter vertieft werden. Hier könnten neue Ansätze untersucht werden, um die Robustheit von Modellen gegenüber Artefakten oder fehlerhaften Daten zu erhöhen.

Schließlich bietet die Verwendung von DA-Fusion zur Darstellung unterschiedlicher Gebrauchszustände von Objekten viele Möglichkeiten zur Weiterentwicklung, etwa durch Begrenzung der zu augmentierenden Bildregion auf einen Teil der Objekte.