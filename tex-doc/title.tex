% !TEX root = Bachelorarbeit Synthetische Daten.tex
\begin{titlepage}
  % Andere Seitenränder als im Rest der Arbeit
  \newgeometry{lmargin=2cm,tmargin=7mm,rmargin=5mm,bmargin=1cm}
  % Hausfarbe der HAW; diese und die folgenden Einstellungen sind lokal und gelten nur innerhalb der Umgebung "titlepage"
  \color{haw}
  % Blocksatz für die Titelseite deaktivieren
  \raggedright
  % Logos rechtsbündig setzen
  \hfill\includegraphics[width=6cm]{logo_haw.jpg}\\
  \vspace{0.5cm}
  \hfill\includegraphics[width=5.6cm]{logo_fraunhofer-ipk.png}\\

  \vspace{5cm}

  % Hausschrift Open Sans der HAW
  \setmainfont{Open Sans}
  % Etwas kleiner als üblich
  \small
  % Fett und in Majuskeln
  \textbf{BACHELORARBEIT}

  \vspace{8mm}

  % Titel der Arbeit
  \begin{minipage}{0.8\linewidth}
    % Haupttitel mit der zweiten Hausschrift der HAW
    \setmainfont{Martel Heavy}
    \LARGE
    Contrastive Learning\\[1mm] % [1mm] für etwas größeren Durchschuss
    mit Stable Diffusion-basierter\\[1mm]
    Datenaugmentation\\[4mm]
    % Untertitel wieder mit der ersten Hausschrift der HAW
    \setmainfont{Open Sans}
    \Large
    Verbesserung der Bildklassifikation\\[1mm]
    durch synthetische Daten\\[1mm]
    % Horizontaler Strich
    \,\rule{11mm}{1.2mm}
  \end{minipage}

  \vspace{9.2mm}

  vorgelegt am 16. September 2024\\
  Paul Hofmann

  \vspace{3cm}

  % Prüfer*innen & Anschriften
  \hspace*{37mm}
  \begin{minipage}{0.5\linewidth}
    \begin{tabular}{@{}ll}
      Erstprüferin: & Prof. Dr. Larissa Putzar\\[-.3mm]
      Zweitprüfer: & Prof. Dr. Jan Neuhöfer \\
    \end{tabular}\\
	
	% Horizontaler Strich
    \,\rule{9mm}{1mm}\\[1.5mm]
	
	% Anschrift HAW
    \textbf{HOCHSCHULE FÜR ANGEWANDTE}\\
    \textbf{WISSENSCHAFTEN HAMBURG}\\
    Department Medientechnik\\
    Finkenau 35\\
    22081 Hamburg
  \end{minipage}
\end{titlepage}
% Setzt Geometrie wieder auf Standardwerte zurück
\restoregeometry

% Seite mit dem Abstract ohne Seitenzahl ausgeben
\thispagestyle{empty}

% Abstract
\section*{Zusammenfassung}

Der Arbeit beginnt mit einer kurzen Beschreibung ihrer zentralen Inhalte, in
der die Thematik und die wesentlichen Resultate skizziert werden.  Diese
Beschreibung muss sowohl in deutscher als auch in englischer Sprache vorliegen
und sollte eine Länge von etwa 150 bis 250 Wörtern haben.  Beide Versionen
zusammen sollten nicht mehr als eine Seite umfassen.  Die Zusammenfassung
dient u.\,a.\ der inhaltlichen Verortung im Bibliothekskatalog.

% In Englisch
{
  \begin{otherlanguage}{english}
    \section*{Abstract}

    The thesis begins with a brief summary of its main contents, outlining the
    subject matter and the essential findings.  This summary must be provided
    in German and in English and should range from 150 to 250 words in length.
    Both versions combined should not comprise more than one page.  Among
    other things, the abstract is used for library classification.
  \end{otherlanguage}
}
