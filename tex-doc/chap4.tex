% !TEX root = Bachelorarbeit Synthetische Daten.tex
\chapter{Ergebnisse}

Dieses Kapitel präsentiert die Ergebnisse der Arbeit. Es wird auf die generierten synthetischen Daten eingegangen und die Trainings- und Testergebnisse der Modelle beschrieben. Anschließend wird die Klassifikations-Performance der Modelle verglichen und die Out-of-Distribution-Detektion analysiert.

\section{Die generierten synthetischen Daten}

% Einleitung/Überblick
% Gemeinsames Stable Diffusion-Finetuning
    % Validation Images; stetiges Verbessern
...

\subsection{In-Distribution}

% Konfiguration von DA-Fusion
% Beispiele
% Menschliche Evaluierung (eigene)
	% Größtenteils überzeugend
	% Teilweise questionable
...

\begin{figure}[]
	\centering
	\includegraphics[width=6cm]{logo_haw}
	\caption{Beispieltext}
\end{figure}

\subsection{Near Out-of-Distribution}

% Konfiguration von DA-Fusion
% Beispiele
% Menschliche Evaluierung (eigene)
...

\section{Trainings- und Testergebnisse mit Supervised Contrastive Leraning}

...

\subsection{Contrastive Pre-Training}

% Nur reale Daten
% Mit ID-Augmentationen
% Mit ID-Augmentationen und OOD-Augmentationen für hard-negative mining
...

\subsection{Lineare Klassifikation}

% Nur reale Daten
% Mit ID-Augmentationen
% Mit ID-Augmentationen und Contrastive Pre-Training mit OOD-Augmentationen
...

\section{Vergleich der Ergebnisse mit und ohne In-Distribution-Augmentationen}

...

\section{Vergleich der Ergebnisse mit und ohne Near Out-of-Distribution-Augmentationen als Hard Negatives}

...
