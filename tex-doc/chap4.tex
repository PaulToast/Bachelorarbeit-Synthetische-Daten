% !TEX root = Bachelorarbeit Synthetische Daten.tex
\chapter{Ergebnisse}

Dieses Kapitel präsentiert die Ergebnisse der Arbeit. Es wird auf die generierten synthetischen Daten eingegangen und die Trainings- und Testergebnisse der Modelle beschrieben. Anschließend wird die Klassifikations-Performance der Modelle verglichen und die Out-of-Distribution-Detektion analysiert.

\section{Die generierten synthetischen Daten}

% Einleitung/Überblick
% Gemeinsames Stable Diffusion-Finetuning
    % Validation Images; stetiges Verbessern
...

\subsection{In-Distribution}

% Konfiguration von DA-Fusion
% Beispiele
% Menschliche Evaluierung (eigene)
	% Größtenteils überzeugend
	% Teilweise questionable
...

\subsection{Near Out-of-Distribution}

% Konfiguration von DA-Fusion
% Beispiele
% Menschliche Evaluierung (eigene)
...

\section{Trainings- und Testergebnisse mit Supervised Contrastive Leraning}

...

\subsection{Contrastive Pre-Training}

% Nur reale Daten
% Mit ID-Augmentationen
% Mit ID-Augmentationen und OOD-Augmentationen für hard-negative mining
...

\begin{figure}
	\centering
	\includegraphics[width=0.5\textwidth]{figure_results_supcon-pre_avg-train-loss.png}%
	\includegraphics[width=0.5\textwidth]{figure_results_supcon-pre_avg-val-loss.png}
	\caption{Beispieltext}
\end{figure}

...

\subsection{Lineare Klassifikation}

% Nur reale Daten
% Mit ID-Augmentationen
% Mit ID-Augmentationen und Contrastive Pre-Training mit OOD-Augmentationen
...

\begin{figure}
	\centering
	\includegraphics[width=0.5\textwidth]{figure_results_supcon-lin_avg-train-loss.png}%
	\includegraphics[width=0.5\textwidth]{figure_results_supcon-lin_avg-val-loss.png}
	\caption{Beispieltext 1}
\end{figure}
\begin{figure}
	\centering
	\includegraphics[width=0.5\textwidth]{figure_results_supcon-lin_avg-train-acc.png}%
	\includegraphics[width=0.5\textwidth]{figure_results_supcon-lin_avg-val-acc.png}
	\caption{Beispieltext 2}
\end{figure}
\begin{figure}
	\centering
	\includegraphics[width=0.5\textwidth]{figure_results_supcon-lin_avg-id-conf.png}%
	\includegraphics[width=0.5\textwidth]{figure_results_supcon-lin_avg-ood-conf.png}
	\caption{Beispieltext 3}
\end{figure}

...

\begin{table}
	\centering
	\begin{tabular}{|c|c|c|c|c|}
		\hline
		\textbf{Versuch} & \textbf{Accuracy (\%)} & \textbf{ID-Confidence} & \textbf{OOD-Confidence} & \textbf{$\Delta$ ID-/OOD-Confidence} \\
		\hline
		1 & 71.4 & 0.69 & 0.62 & 0.07 \\
		2 & 77.5 & 0.76 & 0.65 & 0.11 \\
		3 & 75.3 & 0.40 & 0.35 & 0.05 \\
		\hline
	\end{tabular}
	\caption{Testergebnisse der linearen Klassifikation.}
\end{table}

\section{Vergleich der Ergebnisse mit und ohne synthetische In-Distribution-Daten}

...

\section{Vergleich der Ergebnisse mit und ohne synthetische Near Out-of-Distribution-Daten}

...
