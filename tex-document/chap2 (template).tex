\chapter{Theoretische Grundlagen}

% Hier sieht man, wie mit \ref Bezug genommen wird auf Kapitel, Tabellen und
% Abschnitte.

In diesem Kapitel geht es um die wesentlichen Teile, die eine Abschlussarbeit
haben sollte.  Die Dateien, auf die hier Bezug genommen wird, werden in in
Tabelle~\ref{tab-files} vorgestellt und in Kapitel~\ref{ch-tech} näher
erläutert.  Die Vorlage wurde absichtlich in relativ viele Dateien zerlegt,
um zu demonstrieren, wie man mit dem Befehl \verb|\includeonly| nur Teile der
Arbeit kompilierern kann, um Zeit zu sparen.  Vor dem Druck muss natürlich das
komplette, aus allen Dateien bestehende Dokument kompiliert werden (ggf.\
mehrfach), damit alle Querverweise (siehe Abschnitt~\ref{sec-ref}) korrekt
sind.

% Ein Beispiel für eine Tabelle; die Tabelle selbst wird mit der
% Standardumgebung \tabular gesetzt.  Die Umgebung \table sorgt u.a. für das
% "Gleiten" und die Aufnahme in das Tabellenverzeichnis.
\begin{table}[!ht]
  % Beschriftung der Tabelle
  \caption{Dateien der Vorlage}
  % Zentrieren der Tabelle
  \centering
  \begin{tabular}{ll}
    % unterschiedliche Strichbreiten für "top", "mid" und "bottom"
    \toprule
    Dateiname & Zweck \\
    \midrule
    \texttt{VorlageBA.tex} & Hauptdatei \\
    \texttt{defs.tex} & Laden von Paketen und Setzen von Optionen \\
    \texttt{title.tex} & Titelseite und Abstract \\
    \texttt{toc.tex} & Inhaltsverzeichnis und optionale Verzeichnisse \\
    \texttt{chap1.tex} bis \texttt{chap3.tex} & erstes, zweites und drittes Kapitel \\
    \texttt{appendix.tex} & Literaturverzeichnis, Anhang \\
    & und Eigenständigkeitserklärung \\
    \texttt{demo.bib} & exemplarische Bibliographie \\[3pt]
    \verb|HAW_Marke_RGB_300dpi.jpg| & HAW-Logo für Titelseite \\[3pt]
    \texttt{bitmap.png}, \texttt{vector.pdf} & \\
    und \texttt{euler.py} & Beispieldateien \\
    \bottomrule
  \end{tabular}
  % frei wählbarer Name für \ref
  \label{tab-files}
\end{table}

\section{Titelei}

Mit dem Begriff \textit{Titelei} bezeichnet man im Buchwesen den Teil eines
Buches, der dem eigentlichen Inhalt vorangestellt ist.  In dieser Vorlage
dient der Absatz, den Sie gerade lesen, im Wesentlichen aber nur als Vorwand,
eine weitere Unterebene einzufügen.

\subsection{Titelseite}

Die Titelseite befindet sich in der Datei \texttt{title.tex} und ist der
\href{https://www.haw-hamburg.de/hochschule/hochschuleinheiten/presse-und-kommunikation/corporate-design/}{Vorlage}
der HAW nachempfunden.  Sie müssen natürlich in dieser Datei den Titel, die
Namen und das Datum anpassen.  Außerdem müssen die
\href{https://www.haw-hamburg.de/fileadmin/zentrale_PDF/Zentrale_Dokumente/Corporate_Design/HAW_Schriftenpaket.zip}{Hausschriften}
der HAW \href{https://support.microsoft.com/de-de/office/hinzuf\%C3\%BCgen-einer-schriftart-b7c5f17c-4426-4b53-967f-455339c564c1}{installiert} sein.

\subsection{Abstract}\label{sec-abstract}

Nach der Titelseite folgt der Abstract, der sich ebenfalls in der Datei
\texttt{title.tex} befindet.  Hier ersetzen Sie den Beispieltext durch eine
möglichst aussagekräftige Zusammenfassung Ihrer Arbeit.

\section{Inhaltsverzeichnis und andere Verzeichnisse}\label{sec-lists}

% Hier sieht man ein Beispiel für ein Buchzitat mit einer Kapitelangabe
Nach dem Abstract (siehe Abschnitt~\ref{sec-abstract}) folgt das
Inhaltsverzeichnis, das von \LaTeX\ automatisch erzeugt wird.  Exemplarisch
wird in dieser Vorlage auch gezeigt, wie man Abbildungs- und
Tabellenverzeichnisse generieren könnte; siehe dazu die Datei
\texttt{toc.tex}.  Das ist in der Regel aber nur dann nötig, wenn es in Ihrer
Arbeit sehr viele Abbildungen bzw.\ Tabellen gibt.  Bei Bedarf können auch
Verzeichnisse von Codeblöcken, mathematischen Formeln oder anderen Objekten
erzeugt werden; mehr dazu in \parencite[Kap.\ 12]{voss}.  Hier gilt aber
ebenfalls, dass das normalerweise nicht nötig sein sollte.
Auch ein Glossar oder ein Abkürzungsverzeichnis ist in einer Bachelorarbeit
eher unüblich.  Sprechen Sie ggf.\ mit Ihrer Erstprüferin oder Ihrem
Erstprüfer ab, was wirklich gebraucht wird.

\section{Gliederungsebenen}

Eine Bachelorarbeit sollte in der Regel mit maximal drei Gliederungsebenen
auskommen.  In der Dokumentenklasse der Vorlage entspricht das den Befehlen
\verb|\chapter|, \verb|\section| und \verb|\subsection|, die auch für eine
automatische Aufnahme der jeweiligen Abschnitte ins Inhaltsverzeichnis sorgen.
Weitere Gliederungsebenen verringern typischerweise die Lesbarkeit des
Dokuments.  Falls Sie das bei Ihrer Arbeit trotzdem für nötig halten, sprechen
Sie es vorher mit der Erstprüferin bzw.\ dem Erstprüfer ab.

Auf jeder Gliederungsebene sollte es mindestens zwei Abschnitte derselben
Hierarchie geben, da ansonsten eine Gliederung auf dieser Ebene sinnlos wäre.
Wenn Sie also einen Abschnitt 2.3.1 haben, dann muss es mindestens einen
weiteren Abschnitt 2.3.2 geben; anderenfalls sollte alles unter 2.3 stehen.

% Hier sieht man am Ende des Absatzes, dass man für Auslassungspunkte nicht
% einfach "..." tippt.  Man beachte auch die Tilde davor: das "geschützte
% Leerzeichen".
Außerdem sollten einzelne Abschnitte einen Umfang haben, der die entsprechende
Gliederung rechtfertigt.  Dieser Text, in dem Abschnitte meistens nur aus
wenigen Sätzen bestehen, ist dafür ein schlechtes Beispiel!\footnote{Darum
  wirken die Abstände und Seitenaufteilungen in dieser Vorlage auch
  vergleichsweise unruhig.  Bei längeren Texten wird es besser aussehen.} Es
handelt sich allerdings auch um eine Vorlage und nicht um eine
Bachelorarbeit~\dots

In der Vorlage ist jedes Kapitel in eine eigene Datei ausgelagert.  Beachten
Sie, dass der Befehl \verb|\include| grundsätzlich eine neue Seite anfängt.
Unterabschnitte von Kapiteln müssen daher mit \verb|\input| eingefügt werden,
wenn sie in separaten Dateien liegen sollen.

\section{Literaturverzeichnis}

% Beachten Sie, dass bei Abkürzungen wie "z.B." (siehe unten) oder "u.a." ein
% "kleines" Leerzeichen hinter den ersten Punkt gehört
Die Vorlage enthält ein exemplarisches Literaturverzeichnis, das nach dem
sogenannten APA-Standard formatiert ist und das als Beispiele einige Bücher,
einen Fachartikel und eine Internetquelle umfasst.  Das Literaturverzeichnis
befindet sich am Ende der Arbeit vor dem Anhang.  Falls Ihre Erstprüferin oder
Ihr Erstprüfer ein anderes Format wünscht, ist das mit dem verwendeten Paket
\href{https://www.ctan.org/pkg/biblatex}{Bib\LaTeX} ohne große Probleme zu
realisieren, siehe z.\,B. \parencite[Kap.\ 13]{voss}.  Wichtig ist, dass das
Literaturverzeichnis vollständig ist und Sie nur die Quellen angeben, die Sie
im Rahmen Ihrer Bachelorarbeit wörtlich zitiert oder sinngemäß wiedergegeben
haben.  Literatur, die Sie lediglich zur Vorbereitung genutzt haben, gehört
nicht in das Literaturverzeichnis.\footnote{Wenn Sie die Voreinstellungen der
  Vorlage verwenden, werden aber ohnehin nur die Quellen ins
  Literaturverzeichnis übernommen, die auch zitiert werden.}

Wenn Sie die von Ihnen verwendeten Texte nach dem Muster der Datei
\texttt{demo.bib} eingeben, wird das Literaturverzeichnis automatisch
einheitlich dargestellt und alphabetisch sortiert.  Die Literaturverwaltung
\href{https://de.wikipedia.org/wiki/Citavi}{\textsc{Citavi}}, für die es eine
Hochschullizenz gibt, kann Daten im sogenannten Bib\TeX-Format ausgeben.  (Das
ist das in \texttt{demo.bib} verwendete Format.)

Verwenden Sie wissenschaftliche Quellen.  (Hinweis: Wikipedia gilt nicht als
wissenschaftliche Quelle.)  Wenn die Angabe von Internetquellen unumgänglich
ist, muss das Datum des letzten Aufrufs angegeben werden.  Ein Beispiel dafür
finden Sie in der Vorlage.

Wie man zitiert, wird in Abschnitt~\ref{sec-litref} gezeigt.

\section{Anhang}\label{sec-app}

Falls zu Ihrer Arbeit Datenreihen, Quelltexte, transkribierte Interviews oder
weitere ergänzende Informationen gehören, dann gehören diese in einen Anhang
ganz am Ende.  Ob ein Anhang notwendig ist und welchen Umfang er haben sollte,
sollten Sie vorab mit Ihrer Erstprüferin oder Ihrem Erstprüfer absprechen.
Häufig ist es sinnvoller, die Daten auf einem Datenträger der Arbeit
beizulegen.

Die Vorlage enthält einen exemplarischen Anhang in der Datei
\texttt{appendix.tex}.

\section{Eigenständigkeitserklärung}

Die letzte Seite der Arbeit ist die Eigenständigkeitserklärung, die Sie in der
Datei \texttt{appendix.tex} finden.  Tragen Sie hier den Titel Ihrer Arbeit
sowie den Ort und das Datum ein.  Alle gedruckten Exemplare werden dann
oberhalb des Datums von Ihnen unterschrieben.
