% In dieser Umgebung wird die Titelseite separat vom restlichen Text gesetzt
\begin{titlepage}
  % andere Seitenränder als im Rest der Arbeit
  \newgeometry{lmargin=2cm,tmargin=7mm,rmargin=5mm,bmargin=1cm}
  % die "Hausfarbe" der HAW; diese und die folgenden Einstellungen sind lokal
  % und gelten nur innerhalb der Umgebung "titlepage"
  \color{haw}
  % Blocksatz für die Titelseite deaktivieren
  \raggedright
  % Logos rechtsbündig setzen
  \hfill\includegraphics[width=6cm]{HAW_Marke_RGB_300dpi}\\
  \vspace{0.5cm}
  \hfill\includegraphics[width=5.6cm]{Fraunhofer-IPK_Logo}\\

  \vspace{5cm}

  % Wahl der "Hausschrift" Open Sans der HAW, die als Schrift auf Ihrem
  % Rechner installiert sein muss
  \setmainfont{Open Sans}
  % etwas kleiner als üblich
  \small
  % fett und in Majuskeln
  \textbf{BACHELORARBEIT}

  \vspace{8mm}

  % der Titel der Arbeit als "Seite in der Seite"; natürlich müssen Sie hier
  % Ihren Titel eintragen
  \begin{minipage}{0.8\linewidth}
    % Wahl der zweiten "Hausschrift" der HAW, die ebenfalls auf Ihrem Rechner
    % bereits vorhanden sein muss
    \setmainfont{Martel Heavy}
    % ziemlich große Schrift
    \LARGE
    % [1mm] steht jeweils für einen etwas größeren Durchschuss
    Contrastive Learning\\[1mm]
    mit Stable Diffusion-basierter\\[1mm]
    Datenaugmentation\\[1mm]
    \setmainfont{Open Sans}
    \Large
    Verbesserung der Bildklassifikation\\[1mm]
    durch synthetische Daten\\[1mm]
    % am Ende noch ein waagerechter Strich, das CD will es so...
    \,\rule{11mm}{1.2mm}
  \end{minipage}

  \vspace{1cm}

  % hier korrektes Datum und Ihren Namen eingeben
  vorgelegt am 16. September 2024\\
  Paul Hofmann

  \vspace{3cm}

  % noch eine "Seite in der Seite", etwas nach rechts geschoben
  \hspace*{37mm}
  \begin{minipage}{0.5\linewidth}
    % Namen und Titel der beiden Prüfer eintragen
    \begin{tabular}{@{}ll}
      Erstprüferin: & Prof. Dr. Larissa Putzar\\[-.3mm]
      Zweitprüfer: & Prof. Dr. Jan Neuhöfer \\
    \end{tabular}\\

    % noch ein horizontaler Strich
    \,\rule{9mm}{1mm}\\[1.5mm]

    \textbf{HOCHSCHULE FÜR ANGEWANDTE}\\
    \textbf{WISSENSCHAFTEN HAMBURG}\\
    Department Medientechnik\\
    Finkenau 35\\
    22081 Hamburg
  \end{minipage}
\end{titlepage}
% setzt die Geometrie wieder auf die Standardwerte zurück
\restoregeometry

% für die Seite mit dem Abstract keine Seitenzahl ausgeben
\thispagestyle{empty}

\section*{Zusammenfassung}

% Hier ersetzen Sie bitte die vorhandenen Texte durch Ihre eigenen
% Zusammenfassungen
Der Arbeit beginnt mit einer kurzen Beschreibung ihrer zentralen Inhalte, in
der die Thematik und die wesentlichen Resultate skizziert werden.  Diese
Beschreibung muss sowohl in deutscher als auch in englischer Sprache vorliegen
und sollte eine Länge von etwa 150 bis 250 Wörtern haben.  Beide Versionen
zusammen sollten nicht mehr als eine Seite umfassen.  Die Zusammenfassung
dient u.\,a.\ der inhaltlichen Verortung im Bibliothekskatalog.

% Zum Wechseln der Sprache siehe den Kommentar in chap3.tex
{
  \begin{otherlanguage}{english}
    \section*{Abstract}

    The thesis begins with a brief summary of its main contents, outlining the
    subject matter and the essential findings.  This summary must be provided
    in German and in English and should range from 150 to 250 words in length.
    Both versions combined should not comprise more than one page.  Among
    other things, the abstract is used for library classification.
  \end{otherlanguage}
}
