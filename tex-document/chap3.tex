\chapter{Methodisches Vorgehen}

Dieses Kapitel behandelt die Methoden und Herangehensweisen, die in dieser Arbeit zur Beantwortung der Forschungsfragen angewendet wurden.

Zunächst wird auf den Datensatz eingegangen, der die Grundlage der Arbeit ist und beschrieben, wie daraus … Anschließend wird vorgestellt, wie die beiden Modelle implementiert werden sollen und wie dessen Integration geplant.

Forschungsfragen und Hypothesen:

\textbullet Kann DA-Fusion für den Datensatz des Fraunhofer-IPK überzeugende synthetische Daten generieren?
\textbullet Eignet sich DA-Fusion, um sowohl positiv- als auch negativ-Beispiele für das Contrastive Learning zu generieren?
\textbullet Kann der beschriebene Ansatz eine bessere Generalisierung und Robustheit erzielen als ohne synthetische Daten bzw. als eine naive Verwendung der synthetischen Daten ohne Contrastive Learning?
Starte mit einer Einleitung und dem ersten Punkt zu den Forschungsfragen

\section{Datensatz}

...

\section{Implementierung}

...

\subsection{DA-Fusion}

...

\subsection{Supervised Contrastive Learning}

...

\section{Synthetische Datengenerierung mit DA-Fusion}

...

\section{Trainingsdurchläufe mit Contrastive Learning}

...

\section{Evaluationsmethoden und Metriken}

...

\section{Analyse der Ergebnisse}

...